\chapter{无穷复数项级数}
\begin{introduction}
    \item 复数项级数
    \item 复幂级数
    \item Taylor级数
    \item Laurent级数
    \item 解析延拓
\end{introduction}

\section{复数项级数}
    \subsection{复数项级数}
    复数项级数和实数项级数完全类似,直接快进到审敛法(

    \subsection{复数项级数的审敛定理}
    这里给出复数项级数的审敛定理.
    \begin{enumerate}
        \item 比值判别法\\
                若$\exists N \subset \mathbbm{N}$,对$\forall n > N$,都有
                $|u_n| < v_n$,并且$\sum_{n = 0}^{\infty}v_n$收敛,则$\sum_{n=0}^{\infty}|u_n|$收敛,即$\sum_{n=0}^{\infty}u_n$绝对收敛.
                若$u_n>v_n>0$,且$\sum_{n=0}^{\infty}v_n$发散,则$\sum_{n=0}^{\infty}|u_n|$发散.
        \item 比值判别法\\
                若存在与$n$无关的常数$\rho$,使得$\left|\dfrac{u_{n+1}}{u_n}\right|<\rho<1$,则级数$\sum_{n=0}^{\infty}u_n$绝对收敛;
                若$\left|\dfrac{u_{n+1}}{u_n}\right|>\rho>1$,则级数发散.
    \end{enumerate}

\section{复幂级数}
    \subsection{复幂级数的定义}

    \subsection{复幂级数的敛散性判断}

    \subsection{复幂级数的收敛半径计算}

    \subsection{复幂级数和函数的解析性及其性质}

\section{Taylor级数}
    \subsection{Taylor级数的定义}

    \subsection{常见的Taylor级数}

    \subsection{解析函数的零点孤立性和唯一性}

\section{Laurent级数}
    \subsection{Laurent级数的定义}

    \subsection{Laurent级数补充讨论}

    \subsection{常见的Laurent级数}

    \subsection{奇点的分类}


\section{解析延拓}

    \subsection{解析延拓的定义}

    \subsection{解析延拓的应用}