\chapter{积分变换}

\begin{introduction}
    \item 积分变换简述
    \item Laplace变换
    \item Fourier变换
\end{introduction}

\section{积分变换简述}

    积分变换是求解数学物理问题的重要工具,它是一种把函数$f(t)$经过积分运算变为另一类函数$F(\beta)$的过程,一般表示为:
    \begin{align*}
        F(\beta) = \int_{-\infty}^{\infty} f(t) K(\beta,t) dt
    \end{align*}
    其中,$K(\beta,t)$是一个确定的二元函数,被称为积分变换的核(core)。不同的核与不同的积分区域构成不同的积分变换.

    Laplace变换和Fourier变换是两种常用的积分变换,它们的核分别为$e^{-\beta t}$,$e^{-2\pi i \beta t}$.其他积分变换的核与积分区域可以参考维基百科上的\href{https://zh.wikipedia.org/wiki/%E7%A7%AF%E5%88%86%E5%8F%98%E6%8D%A2}{积分变换表}.


\section{Fourier变换}

\subsection{Fourier变换的定义}

    Fourier变换是由Fourier积分引入的,它表示为:
    \begin{align}
        F(\omega)=\int_{-\infty}^\infty f(x)e^{-i\omega x}
    \end{align}


\section{Laplace变换}

\subsection{Laplace变换的定义}

    


